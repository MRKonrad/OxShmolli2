T\-O\-M\-A\-T\-O (Total Mapping Toolbox) is a C++ library for the calculation of parametric maps in cardiac magnetic resonance imaging (M\-R\-I). As an open source project, T\-O\-M\-A\-T\-O allows transparent and standardized cardiac longitudinal relaxation time (T1) mapping in clinical applications. With C++ implementation, T\-O\-M\-A\-T\-O can easily interface and translate between research software environments, and commercial vendors’ closed-\/source C++ environments on scanners as well as post-\/processing software. To complement the core library implementation, a ready-\/to-\/use command line tool has been provided.

It contains Sh\-M\-O\-L\-L\-I implementation as in \href{https://jcmr-online.biomedcentral.com/articles/10.1186/1532-429X-12-69}{\tt this article}.

\begin{TabularC}{2}
\hline
\rowcolor{lightgray}{\bf System }&\PBS\centering {\bf Status  }\\\cline{1-2}
\-:tomato\-: \href{https://github.com/MRKonrad/tomato_docs}{\tt Tutorial} \-:tomato\-: &\PBS\centering \href{https://github.com/MRKonrad/tomato_docs}{\tt Tomato Docs} \\\cline{1-2}
Code documentation &\PBS\centering \href{https://mrkonrad.github.io/tomato/html/md__r_e_a_d_m_e.html}{\tt Doxygen} \\\cline{1-2}
\href{https://zenodo.org/badge/latestdoi/142749215}{\tt D\-O\-I} &\PBS\centering \href{https://zenodo.org/badge/latestdoi/142749215}{\tt !\mbox{[}D\-O\-I\mbox{]}\mbox{[}zenodo\-\_\-badge\mbox{]}} \\\cline{1-2}
\href{https://travis-ci.org/MRKonrad/tomato}{\tt O\-S\-X/\-Linux build -\/ Travis} &\PBS\centering \href{https://travis-ci.org/MRKonrad/tomato}{\tt !\mbox{[}Travis (.org)\mbox{]}\mbox{[}travis\-\_\-badge\mbox{]}} \\\cline{1-2}
\href{https://ci.appveyor.com/project/MRKonrad/tomato}{\tt Windows build -\/ App\-Veyor} &\PBS\centering \href{https://ci.appveyor.com/project/MRKonrad/tomato}{\tt !\mbox{[}App\-Veyor\mbox{]}\mbox{[}appveyor\-\_\-badge\mbox{]}} \\\cline{1-2}
\href{https://www.codacy.com/app/MRKonrad/tomato?utm_source=github.com&amp;utm_medium=referral&amp;utm_content=MRKonrad/tomato&amp;utm_campaign=Badge_Grade}{\tt Code quality -\/ Codacy} &\PBS\centering \href{https://www.codacy.com/app/MRKonrad/tomato?utm_source=github.com&amp;utm_medium=referral&amp;utm_content=MRKonrad/tomato&amp;utm_campaign=Badge_Grade}{\tt !\mbox{[}Codacy Badge\mbox{]}\mbox{[}codacy\-\_\-badge\mbox{]}} \\\cline{1-2}
\href{https://codecov.io/gh/MRKonrad/tomato}{\tt Test coverage -\/ Codecov} &\PBS\centering \href{https://codecov.io/gh/MRKonrad/tomato}{\tt !\mbox{[}codecov\mbox{]}\mbox{[}codecov\-\_\-badge\mbox{]}} \\\cline{1-2}
\href{https://github.com/MRKonrad/tomato/releases}{\tt Downloads} &\PBS\centering \href{https://github.com/MRKonrad/tomato/releases}{\tt !\mbox{[}Git\-Hub Releases (by Release)\mbox{]}\mbox{[}downloads\-\_\-badge\mbox{]}} \\\cline{1-2}
\end{TabularC}
\subsection*{Tomato\-Open\-Source and Tomato\-Full}

{\bfseries Important}

There are two {\ttfamily Tomato} version available\-: {\ttfamily Tomato\-Open\-Source} compiled with publicly available code and {\ttfamily Tomato\-Full} containing additionally private code used for {\ttfamily Amoeba\-Private\-Nr2} fitting algorithm using Nelder–\-Mead algorithm based on Numerical Recipes. Original Sh\-M\-O\-L\-L\-I (and based on it Tomato) uses code based on Numerical Recipes book. Due to Numerical Recipes licence I cannot share this part of the code online. Please contact me if you would like to use this part of the code. To make up for this limitation we provide a number of alternative fitting procedures.

\subsection*{Contributing and code of conduct}

Please see \hyperlink{md_contributing}{contributing.md} and \hyperlink{md_codeofconduct}{codeofconduct.md}.

\subsection*{Troubleshooting}


\begin{DoxyItemize}
\item missing msvcp140.\-dll Download \href{https://www.microsoft.com/en-us/download/details.aspx?id=48145}{\tt https\-://www.\-microsoft.\-com/en-\/us/download/details.\-aspx?id=48145} as suggested in \href{https://stackoverflow.com/questions/32998902/msvcp140-dll-missing}{\tt https\-://stackoverflow.\-com/questions/32998902/msvcp140-\/dll-\/missing}
\end{DoxyItemize}

\subsection*{Notes}


\begin{DoxyItemize}
\item {\bfseries Continuous integration} was tested \href{https://github.com/MRKonrad/ContinousIntegrationPlayground}{\tt in this repo}.
\item {\bfseries Gitmodules}\-: after {\ttfamily git clone} run {\ttfamily git submodule update -\/-\/init -\/-\/recursive} to get the submodules code.
\item It would be nice to follow https\-://github.com/isocpp/\-Cpp\-Core\-Guidelines/blob/master/\-Cpp\-Core\-Guidelines.\-md \char`\"{}\-Cpp\-Core\-Guidlines\char`\"{} as far as C++98 allows.
\item Maybe I can write a script for github releases with \href{https://developer.github.com/v3/repos/releases/#create-a-release}{\tt github releases R\-E\-S\-T api} and \href{http://www.codingpedia.org/ama/how-to-test-a-rest-api-from-command-line-with-curl/#12_HEAD_requests}{\tt curl R\-E\-S\-T api}?
\item L\-M\-F\-I\-T -\/ somehow windows does not like the static library, so the dynamic is used
\end{DoxyItemize}

\subsection*{T\-O\-D\-O}


\begin{DoxyItemize}
\item add acceptance test for 2param
\item maybe rename calculatormolli to calculator3param?
\item maybe add calculator2param to produce different results?
\item update comments in configuration yaml files
\item re-\/organise start point calculation
\item output S\-N\-R maps
\item make sure S\-N\-R and n\-Amoeba\-Calls is the same as in Shmolli
\item fix download\-I\-T\-K\-\_\-linux\-\_\-osx
\item separate projects for building tomato dependencies (done for lmfit)
\item possibly I could make all Functions\-T1 object static
\item write user-\/friendly build scripts (and use them in C\-I) 
\end{DoxyItemize}